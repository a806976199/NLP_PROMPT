\documentclass[t]{beamer}
\usepackage{CJKutf8}
\usepackage{amsfonts}
    \usepackage{amsmath}
    \usepackage{amssymb}
    \usepackage{amsthm}
    \usepackage{enumerate}
    \usepackage{graphicx}
    \usepackage{layout}
    \usepackage{mathrsfs}
    \usepackage{fancyhdr}
    \usepackage{subfigure}
    \usepackage{tcolorbox}
    \usepackage{tikz-cd}
    \usepackage{color}
    \usepackage{pifont}
    \usepackage{verbatim}
    \usepackage{mathtools}
    \usepackage{float}
    \usepackage{bm}
    \usetheme{AnnArbor}
% \usetheme{Antibes}
\usecolortheme{beaver}
\usepackage{listings}

% 设置JSON样式
\lstdefinestyle{json}{
    basicstyle=\tiny\ttfamily,
    columns=fullflexible,
    showstringspaces=false,
    commentstyle=\color{gray},
    keywordstyle=\color{blue},
    stringstyle=\color{red},
    breaklines=true,
    frame=single,
    captionpos=b,
    aboveskip=10pt,
    belowskip=10pt
}

\lstset{
    language=Python, % 设置代码块语言为Python
    breaklines=true, % 自动换行
    basicstyle=\small\ttfamily, % 设置基本字体样式
    keywordstyle=\bfseries\color{blue}, % 设置关键字样式
    commentstyle=\itshape\color{gray}, % 设置注释样式
    showstringspaces=false, % 不显示字符串中的空格
    frame=single, % 设置代码块边框样式
    numbers=left, % 行号显示在左侧
    numberstyle=\tiny\color{gray}, % 设置行号样式
    stepnumber=1, % 设置行号间隔
    tabsize=4 % 设置制表符宽度
}


% 设置shell样式
\lstdefinestyle{shell}{
    language=bash,
    basicstyle=\tiny\ttfamily,
    columns=fullflexible,
    showstringspaces=false,
    commentstyle=\color{gray},
    keywordstyle=\color{blue},
    stringstyle=\color{red},
    breaklines=true,
    frame=single,
    captionpos=b,
    aboveskip=10pt,
    belowskip=10pt
}

\usepackage{subfigure}

% 添加网址的命令
\usepackage{hyperref}
% 这是一个带链接文本的示例:\href{https://www.example.com}{点击这里访问网站}
% 普通的示例:\url{https://www.example.com}
% 表格
\usepackage{booktabs}
\usepackage{multirow}

% \setbeamertemplate{navigation symbols}{}

\usepackage{textpos}

\newcommand{\dif}{\mathrm{d}}
\newtheorem{thm}{{定理}}

% some common command
\newcommand{\mm}[1]{$ #1$\newline}
% \newcommand{\tuichu}{\Rightarrow}
% \newcommand{\li}[1]{\newline#1}



\newcommand{\analysis}[2]{\forall \mathcal{E}{#1},\exists \delta {#2},s.t.}
\newcommand{\denyanalysis}[2]{\exists \mathcal{E}{#1},\forall \delta {#2},s.t.}
\newcommand{\yield}{\Rightarrow }
\newcommand{\jj}{\newline}
\newcommand{\ff}[1]{$ #1$}   % math environment + newline
\newcommand{\fgn}[1]{\begin{equation}#1\end{equation}  }
\newcommand{\fg}[1]{$$ #1$$}   % math environment + newline 
\newcommand{\pf}{$proof.$\newline}
\newcommand{\ee}{\newline\ff{\Box}\newline}
\newcommand{\fenshi}[2]{\ff{\frac{#1}{#2}}}
\newcommand{\shenlue}{\vdots\jj}
\newcommand{\abs}[1]{{\left \lvert #1 \right\rvert}}
\newcommand{\loge}[1]{In ({#1})}
\newcommand{\logical}[2]{log_{#2}^{#1}}
\newcommand{\summary}[3]{$\sum_{{#1}={#2}}^{#3}  $}
\newcommand{\denjia}[2]{{#1}\Leftrightarrow {#2}}
\newcommand{\jihe}[3]{ {#1}  = \{ {#2} \mid {#3} \} }
\newcommand{\ve}[2]{\left\langle {#1},{#2}\right \rangle}
\newcommand{\dakuohao}[2]{\begin{array}{rcl}{#1}\end{array} \} \Rightarrow{#2}}
\newcommand{\sxb}[3]{#1^{#2}_{#3}}
\newcommand{\sss}[2]{#1^{#2}}
\newcommand{\xxx}[2]{#1_{#2}}
\newcommand{\bri}[1]{\uppercase\expandafter{\romannumeral#1}}
\newcommand{\ri}[1]{\romannumeral#1} 
\newcommand{\polynomial}[8]{#1_{#2}#6^{#7}+#1_{#3}#6^{#8}+...+#1_{#4}#6+#1_{#5} }
\newcommand{\newd}[4]{f[{#1}_{#2},{#4},{#1}_{#3}]}
\newcommand{\lb}[2]{\begin{align*}\begin{split}{#1}\{ {#2}\end{split}\end{align*}}
\newcommand{\tab}[1]{\begin{array}{ll} {#1}\end{array}}


% 向量乘积
\newcommand{\avg}[1]{\left\langle #1 \right\rangle}
% 偏微分方程
\newcommand{\difFrac}[2]{\frac{\dif #1}{\dif #2}}
\newcommand{\pdfrac}[2]{\frac{\partial{#1}}{\partial{#2}}}
% 不同章节
\newcommand{\one}[1]{\section{#1}}
\newcommand{\two}[1]{\subsection{#1}}
\newcommand{\three}[1]{\subsubsection{#1}}
\newcommand{\aone}[1]{\section*{#1}}
\newcommand{\atwo}[1]{\subsection*{#1}}
\newcommand{\athree}[1]{\subsubsection*{#1}}
% 大括号,左右都有
\newcommand{\lbra}[1]{\left\{  {\begin{matrix} #1 \end{matrix}}\right. } 
% 样式 括号前缀 + 括号 
\newcommand{\lbras}[2]{{#1}\left\{ {  {\begin{matrix} #2 \end{matrix}}}\right. } 
\newcommand{\rbra}[1]{ \left.  {\begin{matrix} #1 \end{matrix}} \right\}  } 
% 模长
\newcommand{\distance}[1]{\parallel #1\parallel }
% 等价
\newcommand{\equ}{\Longleftrightarrow }
% 共轭
\newcommand{\cja}[1]{\overline{#1}}
% 两个矩阵,上面是 方框[] 下面是线条| 中间是 无
\newcommand{\mtx}[1]{\begin{matrix}#1\end{matrix} }
\newcommand{\bmtx}[1]{\begin{bmatrix}#1\end{bmatrix} }
\newcommand{\vmtx}[1]{\begin{vmatrix}#1\end{vmatrix} }
% \newcommand{\table}[1]{\begin{array}[lr]{ccc} #1 \end{array}}

%输入普通字符
\newcommand{\ww}[1]{\text{#1}}

% 所有内容 直接头文件搞定
\newcommand{\everything}[1]{\begin{document}\begin{CJK*}{UTF8}{gkai}#1\end{CJK*}\end{document}}


% 存放代码(失败了)
\newcommand{\cccode}[1]{\begin{lstlisting}#1\end{lstlisting}}

% 改变特定行序列
\newcommand{\ttt}{\subsection{}}

% 嵌套序号
\newcommand{\eee}[1]{\begin{enumerate}#1\end{enumerate}}


% 模板里面的一些宏
\newcommand{\pdfFrac}[2]{\frac{\partial #1}{\partial #2}}
\newcommand{\OFL}{\mathrm{OFL}}
\newcommand{\UFL}{\mathrm{UFL}}
\newcommand{\fl}{\mathrm{fl}}
\newcommand{\op}{\odot}
\newcommand{\Eabs}{E_{\mathrm{abs}}}
\newcommand{\Erel}{E_{\mathrm{rel}}}
% 变化颜色
\newcommand{\red}{\textcolor{red}}
\newcommand{\blue}{\textcolor{blue}}
% 注释代码
% \newcommand{\undef}[1]{\iffalse #1 \fi}

% 流程图需要用到的宏包
\usepackage{palatino}
\usepackage{tikz}
\usetikzlibrary{shapes.geometric, arrows}
\tikzstyle{startstop} = [rectangle, rounded corners, minimum width = 2cm, minimum height=1cm,text centered, draw = black, fill = red!40]
\tikzstyle{io} = [trapezium, trapezium left angle=70, trapezium right angle=110, minimum width=2cm, minimum height=1cm, text centered, draw=black, fill = blue!40]
\tikzstyle{process} = [rectangle, minimum width=3cm, minimum height=1cm, text centered, draw=black, fill = yellow!50]
\tikzstyle{decision} = [diamond, aspect = 3, text centered, draw=black, fill = green!30]
% 箭头形式
\tikzstyle{arrow} = [->,>=stealth]
% 4个非常重要 的新命令
\newcommand{\start}[2]{    \node (start) [startstop]{#1};\node (in1) [io, below of = start]{#2};\lin{start}{in1}{}}
\newcommand{\stopp}[3]{\node (out1) [io, below of= #1]{#2};\node (stop) [startstop, below of=out1]{#3};\lin{out1}{stop}{} }
\newcommand{\pro}[6]{    \node (#3) [process, #2 of=#1,xshift=#4 cm]{#5};}
\newpage
\newcommand{\lin}[3]{\draw [arrow] (#1) --node [above] {#3} (#2);}


\begin{document}
\begin{CJK*}{UTF8}{gkai}
% 一般第一页显示PPT标题以及作者信息

% \BackgroundPic{./Screenshot from 2022-04-20 16-31-08.png}

% 增加学校 前面
\addtobeamertemplate{title page}{}{
	\begin{tikzpicture}[remember picture,overlay]
		% \node[yshift=85pt,xshift=50pt]{\includegraphics[height=2cm]{Screenshot from 2022-04-20 16-51-21.png}};
\end{tikzpicture}
}
	% \title{时间序列数据集}
	\title{组会汇报}
	\subtitle {} %不需要
	\author{
		陈钶杰\, \\
		专业:计算数学\,
	} % 显示作者
	% \institute {学院:数学科学学院} % 设置学院机构	
	\date{\today}  % 显示日期
\titlepage

% 设置目录
\begin{frame}{目录}
\frametitle{目录}	
\tableofcontents  % 显示目录
\end{frame}


\section{代码调试}

\begin{frame}
    \frametitle{LSTM模型与自然语言模型进行序列分类对比实验}
    \eee{
        % \item 所选用的数据集:美股实时行情数据,经过筛选以后一共得到774支股票,将所有股票数据合并后再分成训练集和测试集.
        % \item 关于预测的准确度,主要做了以下几个测试,完善一下表格(补充实验,添加单个模型训练的所有数据结果,并进行相应的保存)
        \item 在序列任务上选取出现频率高的作为节点单位进行预测.
        \item 增加TCN,InceptionTime,TSiT模型的序列预测结果.
        % \item 看了关于当前语言模型相关内容
    }
\end{frame}


\subsection{当前序列分类任务汇总结果}

% 直接用excl.xsl表示出来,反正用这个没啥意义了。
\begin{frame}
    \begin{table}[ht]
        \centering
        \caption{准确率结果(均用共享型模型进行测试)}
        \large
        \label{tab:example}
        \begin{tabular}{|c|l|r|c|c|}
        \hline
        模型& 平均准确率\\
        % \hline
        % 随机初始化词向量,单一模型建模 & 27.26\% \\
        % \hline
        % Word2Vec方法,单一模型建模 & {27.36\%}\\
        % 1
        % \hline
        % Word2Vec方法,历史数据点:100,单一模型建模& {27.40\%} \\
        % 8
        % \hline
        % Word2Vec方法,窗口大小为2,单一模型建模 & {27.72\%} \\
        \hline        
        Word2Vec方法,窗口大小为1 & {29.49\%} \\        
        \hline
        Word2Vec方法,窗口大小为2 & {28.90\%} \\
        % \hline
        % Word2Vec方法,窗口大小为3,单一模型建模 & {26.88\%} \\
        \hline
        Word2Vec方法,窗口大小为3& {27.06\%} \\
        \hline
        Word2Vec方法,选择频率较高的节点 & {9.59\%} \\
        \hline
        TCN模型 & {34.52\%} \\
        \hline
        InceptionTime模型 & \red{34.80\%} \\
        \hline
        TSiT模型& {34.71\%} \\
        \hline
        transformer模型 & {34.55\%} \\
        \hline
        chatglm模型 & \blue{32.34\%} \\
        \hline
    \end{tabular}
        \end{table}
\end{frame}

\subsection{模型的构建和选择思路}

\begin{frame}
	\frametitle{}
	\begin{itemize}
    \item     使用频率较高的节点作为基底
    \eee{
        \item 将k线图类别中频率最高的前200个作为基底.\\
        比如将若干个节点组"A,B,D,G,C,...,G,C,G"变成"AB","D","GC",...,"GCG"(个数为25).
        \item 每次根据25个节点来预测右面的一个节点.\\
        比如根据25个节点:"AB","D","GC",...,"GCG",去预测最后一个节点,比如预测结果是"AB",那么就把"A"作为预测的结果.
        \item 但是结果表现并不是很好,可能是由于输入序列的不等长,导致循环神经网络效果变差.
    }    
        \item 在github寻找处理序列分类任务的模型时,发现informer模型基本都是用来做时间序列预测的,发现在序列分类任务上,有如下几个模型效果较好,即TCN模型,InceptionTime模型,它们都用到了卷积网络.并且通过实验测试后,平均准确度在分类任务上确实是最高的.
	\end{itemize}
\end{frame}

\section{相关文献阅读}
    
\subsection{新的神经网络训练方法——MLC}

\begin{frame}
    \frametitle{MLC 是研究人员提出的一种优化程序,旨在通过一系列少样本合成任务来激励系统性}
    \begin{itemize}
        \item 探索如何使用元学习技术来帮助模型更好地理解和利用复合性。这可能涉及到训练模型以学习如何组合简单的元素或任务,以构建更复杂的概念或执行更复杂的任务。这个研究领域可能会涉及到深度学习、强化学习、符号推理和其他机器学习方法的结合,以实现更具通用性和适应性的学习系统。其强调如何使用元学习来使模型能够更好地理解和利用任务的复合性,从而提高学习系统的灵活性和性能。
        \item (数学大模型)DeepMath的效果相当不错,主要是通过高质量数据集的构建,人工标注了几千条现代数学知识问答指令,涵盖了微积分,实分析,复分析,概率论,泛函分析,抽象代数,微分方程,微分几何,拓扑学等多个方向。DeepMath大模型正是基于这个高质量的数据集监督微调llama2语言模型而来。
    \end{itemize}
\end{frame}

% \subsection{下一步的计划}
% \begin{frame}
% 	\frametitle{下一步计划及相关问题}	
% 	\begin{itemize}
%         \item 寻找其他相关模型进行比对?
%         % \item 提取模型中的注意力权重,查看模型对于输入信息的处理细节
%         % \eee{
%         %     \item 如何解决经过规范化以后数值比较接近的问题?
%         % }
% 	\end{itemize}
% \end{frame}

% 结束语
\section{}
\begin{frame}
	\frametitle{}
	\begin{center}
		\Huge{谢谢老师和同学们的聆听!}
	\end{center}
\end{frame}

\end{CJK*}
\end{document}
