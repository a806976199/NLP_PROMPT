\documentclass[a4paper,oneside]{article}
\usepackage{geometry}
\geometry{margin=1.5cm, vmargin={0pt,1cm}}
\setlength{\topmargin}{-1cm}
\setlength{\paperheight}{29.7cm}
\setlength{\textheight}{25.3cm}
\usepackage{CJKutf8}
%\usepackage{ctex}
 \usepackage{amsfonts}
\usepackage{amsmath}
\usepackage{amssymb}
\usepackage{amsthm}
\usepackage{enumerate}
\usepackage{graphicx}
\usepackage{multicol}
\usepackage{fancyhdr}
\usepackage{layout}
\usepackage{mathdots}
% 代码需要用到的宏包
\usepackage{listings} %用来写代码
\usepackage{xcolor} %代码高亮
\lstset{language=C++}
\lstset{
    numbers=left, 
    numberstyle= \tiny, 
    keywordstyle= \color{ blue!70},
    commentstyle= \color{red!50!green!50!blue!50}, 
    frame=shadowbox, % 阴影效果
    rulesepcolor= \color{ red!20!green!20!blue!20} ,
    % escapeinside=``, % 英文分号中可写入中文
    escapeinside=(), % 英文分号中可写入中文
    xleftmargin=2em, aboveskip=1em,
    framexleftmargin=2em
 } % 代码设置
%  具体使用的例子
% \noindent 代码如下: 
% \begin{lstlisting}
% \end{lstlisting}

% 多文件使用latex需要用到
\usepackage{subfiles} % Best loaded last in the preamble

 
% 画图学习使用的宏包
\usepackage{tikz}  %画图包
\usetikzlibrary{arrows.meta}%画箭头用的包
\newcommand{\point}[2]{\coordinate (o) at(#2) node[] at (o) {$#1$}; }
% \point{O}{0.5,0.5}
% \large \draw[help lines,step = 0.2] (0,0) grid (1,1); %辅助线格子
% \draw[-latex] (0,0) -- (1.1,0);
% \draw[-latex] (0,0) -- (0,1.1);%实心箭头    
% \draw[green] (0.5,0.5) circle (0.12);%圆
% \draw[yellow] (0,0) rectangle (1,1);%矩形 



% some common command
\newcommand{\mm}[1]{$ #1$\newline}
% \newcommand{\tuichu}{\Rightarrow}
% \newcommand{\li}[1]{\newline#1}



\newcommand{\analysis}[2]{\forall \mathcal{E}{#1},\exists \delta {#2},s.t.}
\newcommand{\denyanalysis}[2]{\exists \mathcal{E}{#1},\forall \delta {#2},s.t.}
\newcommand{\yield}{\Rightarrow }
\newcommand{\jj}{\newline}
\newcommand{\ff}[1]{$ #1$}   % math environment + newline
\newcommand{\fgn}[1]{\begin{equation}#1\end{equation}  }
\newcommand{\fg}[1]{$$ #1$$}   % math environment + newline 
\newcommand{\pf}{$proof.$\newline}
\newcommand{\ee}{\newline\ff{\Box}\newline}
\newcommand{\fenshi}[2]{\ff{\frac{#1}{#2}}}
\newcommand{\shenlue}{\vdots\jj}
\newcommand{\abs}[1]{{\left \lvert #1 \right\rvert}}
\newcommand{\loge}[1]{In ({#1})}
\newcommand{\logical}[2]{log_{#2}^{#1}}
\newcommand{\summary}[3]{$\sum_{{#1}={#2}}^{#3}  $}
\newcommand{\denjia}[2]{{#1}\Leftrightarrow {#2}}
\newcommand{\jihe}[3]{ {#1}  = \{ {#2} \mid {#3} \} }
\newcommand{\ve}[2]{\left\langle {#1},{#2}\right \rangle}
\newcommand{\dakuohao}[2]{\begin{array}{rcl}{#1}\end{array} \} \Rightarrow{#2}}
\newcommand{\sxb}[3]{#1^{#2}_{#3}}
\newcommand{\sss}[2]{#1^{#2}}
\newcommand{\xxx}[2]{#1_{#2}}
\newcommand{\bri}[1]{\uppercase\expandafter{\romannumeral#1}}
\newcommand{\ri}[1]{\romannumeral#1} 
\newcommand{\polynomial}[8]{#1_{#2}#6^{#7}+#1_{#3}#6^{#8}+...+#1_{#4}#6+#1_{#5} }
\newcommand{\newd}[4]{f[{#1}_{#2},{#4},{#1}_{#3}]}
\newcommand{\lb}[2]{\begin{align*}\begin{split}{#1}\{ {#2}\end{split}\end{align*}}
\newcommand{\tab}[1]{\begin{array}{ll} {#1}\end{array}}

%注释代码
\newcommand{\undef}[1]{\iffalse #1 \fi}

% 微分
\newcommand{\dif}{\mathrm{d}}
% 向量乘积
\newcommand{\avg}[1]{\left\langle #1 \right\rangle}
% 偏微分方程
\newcommand{\difFrac}[2]{\frac{\dif #1}{\dif #2}}
\newcommand{\pdfrac}[2]{\frac{\partial{#1}}{\partial{#2}}}
% 不同章节
\newcommand{\one}[1]{\section{#1}}
\newcommand{\two}[1]{\subsection{#1}}
\newcommand{\three}[1]{\subsubsection{#1}}
\newcommand{\aone}[1]{\section*{#1}}
\newcommand{\atwo}[1]{\subsection*{#1}}
\newcommand{\athree}[1]{\subsubsection*{#1}}
% 大括号,左右都有
\newcommand{\lbra}[1]{\left\{  {\begin{matrix} #1 \end{matrix}}\right. } 
% 样式 括号前缀 + 括号 
\newcommand{\lbras}[2]{{#1}\left\{ {  {\begin{matrix} #2 \end{matrix}}}\right. } 
\newcommand{\rbra}[1]{ \left.  {\begin{matrix} #1 \end{matrix}} \right\}  } 
% 模长
\newcommand{\distance}[1]{\parallel #1\parallel }
% 等价
\newcommand{\equ}{\Longleftrightarrow }
% 共轭
\newcommand{\cja}[1]{\overline{#1}}
% 两个矩阵,上面是 方框[] 下面是线条| 中间是 无
\newcommand{\mtx}[1]{\begin{matrix}#1\end{matrix} }
\newcommand{\bmtx}[1]{\begin{bmatrix}#1\end{bmatrix} }
\newcommand{\vmtx}[1]{\begin{vmatrix}#1\end{vmatrix} }
% \newcommand{\table}[1]{\begin{array}[lr]{ccc} #1 \end{array}}

%输入普通字符
\newcommand{\ww}[1]{\text{#1}}

% 所有内容 直接头文件搞定
\newcommand{\everything}[1]{\begin{document}\begin{CJK*}{UTF8}{gkai}#1\end{CJK*}\end{document}}


% 存放代码(失败了)
\newcommand{\cccode}[1]{\begin{lstlisting}#1\end{lstlisting}}

% 改变特定行序列
\newcommand{\ttt}{\subsection{}}

% 嵌套序号
\newcommand{\eee}[1]{\begin{enumerate}#1\end{enumerate}}


% 模板里面的一些宏
\newcommand{\pdfFrac}[2]{\frac{\partial #1}{\partial #2}}
\newcommand{\OFL}{\mathrm{OFL}}
\newcommand{\UFL}{\mathrm{UFL}}
\newcommand{\fl}{\mathrm{fl}}
\newcommand{\op}{\odot}
\newcommand{\Eabs}{E_{\mathrm{abs}}}
\newcommand{\Erel}{E_{\mathrm{rel}}}
% 变化颜色
\newcommand{\red}{\textcolor{red}}
\newcommand{\blue}{\textcolor{blue}}



% 流程图需要用到的宏包
\usepackage{palatino}
\usepackage{tikz}
\usetikzlibrary{shapes.geometric, arrows}
\tikzstyle{startstop} = [rectangle, rounded corners, minimum width = 2cm, minimum height=1cm,text centered, draw = black, fill = red!40]
\tikzstyle{io} = [trapezium, trapezium left angle=70, trapezium right angle=110, minimum width=2cm, minimum height=1cm, text centered, draw=black, fill = blue!40]
\tikzstyle{process} = [rectangle, minimum width=3cm, minimum height=1cm, text centered, draw=black, fill = yellow!50]
\tikzstyle{decision} = [diamond, aspect = 3, text centered, draw=black, fill = green!30]
% 箭头形式
\tikzstyle{arrow} = [->,>=stealth]
% 4个非常重要 的新命令
\newcommand{\start}[2]{    \node (start) [startstop]{#1};\node (in1) [io, below of = start]{#2};\lin{start}{in1}{}}
\newcommand{\stopp}[3]{\node (out1) [io, below of= #1]{#2};\node (stop) [startstop, below of=out1]{#3};\lin{out1}{stop}{} }
\newcommand{\pro}[6]{    \node (#3) [process, #2 of=#1,xshift=#4 cm]{#5};}
\newpage
\newcommand{\lin}[3]{\draw [arrow] (#1) --node [above] {#3} (#2);}



% 显示目录到几级标题
\setcounter{tocdepth}{1}



\begin{document}
% 页眉页脚
\pagestyle{fancy}
\fancyhead{}
\lhead{NAME (ChenKejie 22135030)}
\chead{Numerical Analysis homework \#}
\rhead{Date 2022.3}
% 中文
\begin{CJK*}{UTF8}{gkai}

%标题,文章开头
\title{project2实验报告}
\author{陈钶杰\, \\
  学号: 22135030\\}
\date{\today}
\maketitle
\newpage

% 正式内容
\section*{使用langchain}

\eee{
  \item 
}


\end{CJK*}\end{document}




% 这是文件标题和文件名
% \pagestyle{fancy}
% \fancyhead{}
% \lhead{NAME (put your uID here)}
% \chead{Numerical Analysis homework \#?}
% \rhead{Date}

% 模板格式作业
% \section*{
%         \bri{1}
% }
% \subsection*{\bri{1}}









% \subsection{}
% \subsubsection{模板:\texttt{}}
%     % 作用
% \subsubsection{功能}
% \begin{enumerate}
%     \item 
% \end{enumerate}
% \subsubsection{数据成员}
% \begin{enumerate}
%     \item 
% \end{enumerate}
% \subsubsection{函数}
% \begin{enumerate}
%     \item 
% \end{enumerate}


% \begin{tikzpicture}[node distance=2cm]
%     \start{input}{输入半径等}
%     \pro{in1}{below}{pro1}{0}{first}

%     \pro{pro1}{below}{pro2}{0}{fdjasl}
    
%     \lin{in1}{pro1}{}            
%     \lin{pro1}{pro2}{}            

%     \stopp{pro2}{aa}{bb}    
%     \lin{pro2}{out1}{}    


%     \pro{pro1}{right}{pro1}{2}{c5}    
    
%     \pro{pro1}{below}{pro2}{0}{new}        

%     \lin{in1}{pro1}{}                
%     \lin{pro1}{pro2}{}                

%     \pro{pro1}{right}{pro1}{2}{c7}    

%     \pro{pro1}{below}{pro2}{2}{cccdfkslfjdl}

%     \lin{in1}{pro1}{}                        
%     \lin{pro1}{pro2}{}                            

%     \stopp{pro2}{aa}{bb}
    
%     \lin{pro2}{out1}{}    
% \end{tikzpicture}
